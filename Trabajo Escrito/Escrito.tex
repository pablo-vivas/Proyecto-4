\documentclass[12pt,a4paper]{article}
\usepackage[utf8]{inputenc}
\usepackage[spanish]{babel}
\usepackage{amsmath}
\usepackage{amsfonts}
\usepackage{amssymb}
\usepackage{makeidx}
\usepackage{graphicx}
\usepackage{lmodern}
\usepackage{kpfonts}
\usepackage{apacite}
\usepackage{multirow}
\usepackage{multirow}
\usepackage{graphicx}
\setlength{\parindent}{0em}
\usepackage[left=3cm,right=3cm,top=2cm,bottom=2cm]{geometry}
\author{Pablo Vivas Corrales\footnote{\textit{Maestría Académica en Estadística. Universidad de Costa Rica}}\\\textit{pablo.vivas@ucr.ac.cr}}
\title{Análisis espacial del dengue en Costa Rica}
\date{16 de diciembre de 2019}
\begin{document}
\maketitle
\begin{abstract}
\noindent
La posición geográfica de Costa Rica posibilita al vector \textit{Aedes} la propagación de la enfermedad conocida como dengue. Solo en el año 2019 la sección de vigilancia del Ministerio de Salud reportó más de 8000 mil casos de esta enfermedad, siendo el cantón de Sarapiquí de Heredia un valor extremo (1816 casos registrados). Con esos datos e información obtenida del Instituo Nacional de Estadistica y Censos (INEC) se realizó un análisis espacial con los siguientes objetivos: (1) identificar la existencia de agrupaciones espaciales en los cantones de Costa Rica (2) identificar indicadores que están asociados con la aparición del dengue y (3) calcular el riesgo que tienen los habitantes de los cantones costarricenses de tener esta enfermedad. Se hace uso de la prueba I de Moran para cuantificar la autocorrelación espacial. Asimismo se ajusta un modelo lineal, dos modelos autoregresivos (SAR \& CAR) y un modelo lineal generalizado con un término no paramétrico, siendo este último el que posee un $R^{2}$ más elevado (93\%). Se utilizan técnicas propias del análisis epidemiológico, como el mapa de probabilidades de Choynowski y el cálculo de riesgos relativos, donde resultan significativos los cantones de: Sarapiquí, Guácimo, La Cruz, Turrialba, Atnas, Montes de Oro, Garabito \& Pococí. Se concluye con un análisis empírico bayesiano.\\
\textbf{Palabras clave:} \textit{Dengue, Cantones de Costa Rica, Modelos Autoregresivos \& Análisis Epidemiológico} 
\end{abstract}
\section{Introducción}

El dengue es una enfermedad viral transmitida por mosquitos que se ha propagado rápidamente en todas las regiones de la Organización Mundial de la Salud (OMS) en los últimos años. El virus del dengue es transmitido por mosquitos hembras, principalmente de la especie \textit{Aedes aegypti}. Este mosquito también es vector de virus como chikungunya, fiebre amarilla y zika. El dengue está muy extendido en las regiones tropicales y subtropicales \cite{OMS}. La incidencia del dengue ha crecido dramáticamente en todo el mundo en las últimas décadas. La gran mayoría de los casos son leves y autogestionados, y por lo tanto, el número real de casos de dengue no se informa. Muchos casos también se diagnostican erróneamente como otras enfermedades febriles \cite{Waggoner2016}.\\

El dengue causa un amplio espectro de enfermedades. Esto puede variar desde una enfermedad subclínica (las personas pueden no saber que incluso están infectadas) hasta síntomas severos similares a la gripe en las personas infectadas. Aunque es menos común, algunas personas desarrollan dengue severo, que puede ser cualquier cantidad de complicaciones asociadas con sangrado severo, deterioro de órganos y / o pérdida de plasma. El dengue grave tiene un mayor riesgo de muerte cuando no se maneja adecuadamente. Hoy en día, el dengue grave afecta a la mayoría de los países asiáticos y latinoamericanos y se ha convertido en una de las principales causas de hospitalización y muerte entre niños y adultos en estas regiones.



El dengue es una enfermedad aguda febril, producida por un virus ARN de la familia \textit{Flaviridae}, cuyo único reservorio es el hombre. Existen 4 serotipos distintos DEN- 1, DEN- 2, DEN- 3 y DEN- 4. Es más predominante en las regiones tropicales. El virus se transmite por la picadura de la hembra del mosquito \textit{Aedes sp.} Ésta adquiere la infección al alimentarse de un paciente en fase virémica. El virus se multiplica y alcanza las glándulas salivares de la mosquito hembra, donde se mantiene de por vida, por lo que puede infectar a varias personas. Existen varios tipos de Aedes: Ae. aegypti, Ae. albopictus, Ae. meiovittatus, Ae. scutellaris .etc. El más importante es Ae. aegypti, que se alimenta principalmente sangre humana y lo hace de día. El virus es altamente transmisible cuando la infestación por el vector es alta, lo que puede producir epidemias de dengue con alta morbilidad y mortalidad, en su forma grave. La infección que produce resulta en un amplio espectro de presentaciones clínicas, que van desde formas asintomáticas, indiferenciadas y leves hasta cuadros graves con compromiso vascular, coagulación y órganos blancos.Puede haber transmisión por la picadura directa del mosquito, vía vertical (madre-hijo,tercer trimestre de embarazo) o vía transfusional \cite{CajaCostarricensedelSeguroSocial2013}.
\section{Métodos}

Los datos utilizados utilizados para realizar el análisis espacial del dengue en Costa Rica provienen de dos fuentes. En primer lugar, de la sección de vigilancia del Ministerio de Salud se obtiene la información de los casos de dengue para el 2019 por cantón. Los paquetes de R \cite{R} de readxl \cite{readxl}, sp \cite{sp}, sf \cite{sf}, tidyverse \cite{tidy}, rgdal \cite{rdal}, RColorBrewer \cite{Rcolor}, spdep \cite{sp}, tmap \cite{tmap}, tmaptools \cite{tmaptools}, spatialreg \cite{sp}, epitools \cite{epitools}, DCluster \cite{dcluster}, plotrix \cite{plotrix}, MASS \cite{MASS} \& mgcv \cite{mgcv}.
\section{Resultados}

\begin{figure}[hbtp]
\centering
\includegraphics[width=.48\textwidth]{F11.pdf}
\includegraphics[width=.48\textwidth]{F12.pdf}
\caption{Tasa de dengue (100.000 habitantes) por cantón, 2019.}
\end{figure}

\begin{figure}[hbtp]
\centering
\includegraphics[width=.48\textwidth]{F21.pdf}
\includegraphics[width=.48\textwidth]{F22.pdf}
\caption{Métodos de vecinos: Reina y Knn(4)}
\end{figure}

\begin{figure}[hbtp]
\centering
\includegraphics[width=.48\textwidth]{F31.pdf}
\includegraphics[width=.48\textwidth]{F32.pdf}
\caption{Casos de influencia}
\end{figure}

\begin{figure}[hbtp]
\centering
\includegraphics[scale=0.75]{F4.pdf}
\caption{Residuales de los modelos}
\end{figure}

\begin{figure}[hbtp]
\centering
\includegraphics[scale=0.5]{F5.pdf}
\caption{Riesgo relativo}
\end{figure}

\begin{figure}[hbtp]
\centering
\includegraphics[scale=0.75]{F6.pdf}
\caption{Intervalos de confianza}
\end{figure}

\begin{table}[h]
\centering
\begin{tabular}{cccc}
\hline
\multirow{2}{*}{Vecinos} & \multicolumn{3}{l}{Matriz de pesos}\\ \cline{2-4} 
&W&B&S\\ \hline
Reina&0,009&0,007&0,006\\
Torre&0,011&0,008&0,008\\
Knn(2)&0,018&0,018&0,018\\
Knn(4)&0,026&0,026&0,026\\ \hline
\end{tabular}
\end{table}

\section{Conclusiones}

\section{Anexos}

\begin{figure}[hbtp]
\centering
\includegraphics[width=.48\textwidth]{FA1.pdf}
\includegraphics[width=.48\textwidth]{FA2.pdf}
\caption{Casos de dengue (100.000 habitantes) por cantón, 2019.}
\end{figure}

\begin{figure}[hbtp]
\centering
\includegraphics[width=.48\textwidth]{FA3.pdf}
\includegraphics[width=.48\textwidth]{FA4.pdf}
\includegraphics[width=.48\textwidth]{FA5.pdf}
\includegraphics[width=.48\textwidth]{FA6.pdf}
\caption{Estadísticos descriptivos}
\end{figure}

\begin{figure}[hbtp]
\centering
\includegraphics[scale=0.75]{FA7.pdf}
\caption{Distintas pruebas de Moran}
\end{figure}

\begin{figure}[hbtp]
\centering
\includegraphics[scale=0.75]{FA8.pdf}
\caption{Valores Observados vs Esperados}
\end{figure}

\begin{figure}[hbtp]
\centering
\includegraphics[scale=0.75]{FA9.pdf}
\caption{Mapa de probabilidad de Chownoysky}
\end{figure}

\begin{figure}[hbtp]
\centering
\includegraphics[scale=0.75]{FA10.pdf}
\caption{Distintas pruebas de Moran}
\end{figure}

%%Bibliografía
\bibliographystyle{apacite}
\bibliography{Referencias}
\end{document}


